\documentclass[10pt,twocolumn,letterpaper]{article}
\def\code#1{\texttt{#1}}
\usepackage{natbib}
%% Language and font encodings
\usepackage[english]{babel}
\usepackage[utf8x]{inputenc}
\usepackage[T1]{fontenc}
\usepackage{float}
%% Sets page size and margins
\usepackage[a4paper,top=3cm,bottom=2cm,left=3cm,right=3cm,marginparwidth=1.75cm]{geometry}
%% More packages
\usepackage{amsmath}
\usepackage{graphicx}
\usepackage[colorinlistoftodos]{todonotes}
\usepackage[colorlinks=true, allcolors=blue]{hyperref}
\usepackage{fancyhdr}
\pagestyle{fancy}
\fancyhead[R]{\normalfont \normalsize \textsc{Ms HPC-AI - Big Data Project}}

\fancypagestyle{firstpage}
{
    \fancyhead[L]{ \includegraphics[width=4cm]{fig/Logo_Mines_ParisTech.svg.png}}  
    \fancyhead[R]{\normalfont \normalsize \textsc{Mines ParisTech - Ms HPC-AI}}  
}

\bibliographystyle{unsrt}
%% Title
\title{
		\huge Data Visualitation Project :\\ Ray Tracing with VTK \\
}
\usepackage{authblk}
\author{Aymeric MILLAN \& Arthur VIENS}
\affil{Lecture given by Julien Wintz}
\date{11th of March 2022}
\begin{document}

\setlength\headheight{26pt}
\lhead{\includegraphics[width=4cm]{fig/Logo_Mines_ParisTech.svg.png}}

\maketitle

%%%%%%%%%%%%%%%%%%%%%%%%%%%%%%%%%%%%%%%%%%%%%%%%%%%%%%%%%%%%%%%%%%%%%%%%%%%%%%%%
%%%%%%%%%%%%%%%%%%%%%%%%%%%%%%%%%%%%%%%%%%%%%%%%%%%%%%%%%%%%%%%%%%%%%%%%%%%%%%%%
%% ABSTRACT
\thispagestyle{firstpage}
\begin{abstract}
This documents contains the presentation of our work on implementing a
RayTracing algorithm using the VTK library and Python's bindings.

First, we are going to do a brief sate of the art on raytracing, then we will
explain our implementation logic and technical choices. To conclude, we will
show some images rendered by our application.

Difficultés rencontrées :

Learning VTK's logic and syntax was a difficulty

\end{abstract}

%%%%%%%%%%%%%%%%%%%%%%%%%%%%%%%%%%%%%%%%%%%%%%%%%%%%%%%%%%%%%%%%%%%%%%%%%%%%%%%%
%%%%%%%%%%%%%%%%%%%%%%%%%%%%%%%%%%%%%%%%%%%%%%%%%%%%%%%%%%%%%%%%%%%%%%%%%%%%%%%%
%% RECUPERER LE CODE
\section{Running the code}
The source code is available on
\href{https://git.sophia.mines-paristech.fr/aymeric.millan/visu-raytracing}{Mines' GitLab server}.
You can use the \code{environment.yml} file to setup your Python environment
with conda.

To run the app, just launch the \code{main.py} file and let the UI guide you !

%%%%%%%%%%%%%%%%%%%%%%%%%%%%%%%%%%%%%%%%%%%%%%%%%%%%%%%%%%%%%%%%%%%%%%%%%%%%%%%%
%%%%%%%%%%%%%%%%%%%%%%%%%%%%%%%%%%%%%%%%%%%%%%%%%%%%%%%%%%%%%%%%%%%%%%%%%%%%%%%%
%% STATE OF ART
\section{State of the art}

    Raytracing il ya ca ca ca


    % \begin{figure}[H]
    %       \centering
    %       \caption{Dask output interface}
    %     \includegraphics[width=0.45\textwidth]{fig/dask_interface.png}
    % \end{figure}
    %%%%%%%%%%%%%%%%%%%%%%%%%%%%%
    %%%%%TITLE
    \subsection{1er truc}

Le 1er truc qu'on a trouvé sur l'algo


%%%%%%%%%%%%%%%%%%%%%%%%%%%%%%%%%%%%%%%%%%%%%%%%%%%%%%%%%%%%%%%%%%%%%%%%%%%%%%%%
%%%%%%%%%%%%%%%%%%%%%%%%%%%%%%%%%%%%%%%%%%%%%%%%%%%%%%%%%%%%%%%%%%%%%%%%%%%%%%%%
%% IMPLEMENTATION LOGIC
\section{Implementation logic}

    %%%%%%%%%%%%%%%%%%%%%%%%%%%%%
    %%%%%QT
    \subsection{Python's VTK bindings}

Blabla why we used python, Blabla.
Finir sur une phrase qui dit qu'on a utilisé les bindings avec qt..

    %%%%%%%%%%%%%%%%%%%%%%%%%%%%%
    %%%%%QT
    \subsection{User Interface : Qt Designer}

    We decided to use Qt (with either PySide or PyQt as a wrapper) for the User
interface, this way, we don't have to make complex documentation about what
keyboard shortcut to use to interact with ou VTK panel. 

PICTURE OF SLIDERS
% \begin{figure}[H]
%     \centering
%     \caption{Models' scores and predictions}
%     \includegraphics[width=0.45\textwidth]{fig/scores_models_spark.png}
% \end{figure}

To build the UI we used \href{https://doc.qt.io/qt-5/qtdesigner-manual.html}{Qt Designer},
which generated a \code{.ui} file. This file is compiled at the start of our
program into a \code{.py} file which is the corresponding Qt Windows, with all
sizes and objects' labels automatically set. We do not have to touch this file
once it is generated.This method allowed us to make a more complex UI to give
the user more interaction with our VTK environment.

In the code, we had to make a binding between the objects' methods (e.g.,
\code{ValueChanged} for a Slider) and Python functions taht interacts directly with
our VTK actors and/or sources.

%%%%%%%%%%%%%%%%%%%%%%%%%%%%%%%%%%%%%%%%%%%%%%%%%%%%%%%%%%%%%%%%%%%%%%%%%%%%%%%%
%%%%%%%%%%%%%%%%%%%%%%%%%%%%%%%%%%%%%%%%%%%%%%%%%%%%%%%%%%%%%%%%%%%%%%%%%%%%%%%%
%% CONCLUSION
\section*{Conclusions}

VTK c'est une usine à gaz. C'est fort quand meme on peut render nimporte ou avec
nimporte quel DIrectX, OpenGL, etc... Mais la portabilité a un cout : la complexité

Les bindings python sont efficaces

Peut être faire en C++ si cetait a refaire

Toujours plus d'améliorations possible, surtout au niveau e l'interface
(bouger les objets avec la souris, image de prérendu avant de save un png, etc...)

Amélioration : Séparer en plusieurs classes python propre au liieu d'un gros fichier

\bibliography{bibliography}
\end{document}
